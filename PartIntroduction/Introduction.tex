% !TEX root = ../Thesis.tex
\chapter{Introduction and motivation}
\label{sec:Introduction}

The Large Hadron Collider or LHC, is the most powerful collider in the world and gives scientists a probe to study the universe at an unprecedented energy level. The ATLAS experiment is a general-purpose detector serviced by the LHC designed to record and measure every aspect of the outgoing spray of particles resulting from the colliding LHC beams. High energy research can be divided into several categories: testing the established theory known as the Standard Model of particle (SM), improvement of previously measured parameters and the search for new physics. The SM has stood the test of time and rigorous experimental testing. A crucial part of the theory, the Higgs mechanism, was experimentally validated in 2012 when the ATLAS and CMS experiments independently confirmed the production of Higgs boson in LHC collisions.

Top quark physics concerns itself with the study of the heaviest known quark described by the SM. Due to its large mass the top quark does not bind to other quarks to form a composite particle known as a hadron. The top is the only quark that can therefore be studied directly on its own. In addition, the mass of the top quark is a parameter the SM and many Beyond the Standard Model (BSM) theories.

Due to the large centre of mass energy at the LHC, top quarks are produced in large quantities allowing for detailed studies of many top quark properties. Top quarks can be produced either in single-top events or, more likely at the LHC, into a pair with one top and one antitop (\ttbar\ pair). The top quark decays overwhelmingly into a $W$ boson and a $b$-quark. Subsequently the $W$ can decay leptonically, into a lepton and lepton neutrino; or hadronically, into a pair of quarks. Top quark pair events are classified into three groups depending on the manner in which the $W$ bosons decay: ``all-hadronic'', where both $W$ bosons decay hadronically; ``dilepton'', where both bosons decay into leptons; and finally, ``lepton plus jets'' where one boson decays leptonically and the other hadronically.

The $b$-quark binds with other quarks to form hadrons. This hadron then decays into a collimated shower of particles known as a jet. Identification of these $b$-jets is an important part of any top quark analysis and there are several methods of $b$-tagging in use. The soft muon tagger (SMT tagger), with which this thesis concerns itself, is one such $b$-tagger. $b$-hadrons can decay so as to produce a low momentum muon (also known as a soft muon) which then emerges buried within the subsequent jet. The SMT tagger uses the quality of the reconstruction of so-called combined muons, which rely on both inner detector and muon spectrometer information for reconstruction. The quality of the matching between the inner detector and muon spectrometer tracks is encapsulated in the $\chi^{2}$ of the match. Muon reconstruction and the SMT tagger are described in more detail in Section~\ref{sec:DetectorSLT}.

Measurement of the top quark pair production probability, denoted by the cross section $\sigma_{\ttbar}$, is an important early measurement to make. In particular as the cross section depends on the centre of mass energy of the collision, such a measurement tests the predictive power of the Standard Model at an energy level never studied before. Any new physics processes which share the same signature as \ttbar\ production will result in an excess in the cross section above the theoretically measured value.

An example of new physics include theories that posit the existence of a very heavy boson known as the $Z'$. This boson would preferentially decay to a \ttbar\ pair where each top quark has a large amount of momentum. 

In this thesis the SMT tagger is calibrated and used as part of a cross section measurement and its performance is evaluated in searching for high momentum tops emerging from $Z'$ decays. Measuring the top quark pair production cross section using the SMT tagger is of interest as it tests a different aspect of theory, namely the description of semileptonic $b$-decays, compared to lifetime-based taggers. Such a measurement was carried out and is detailed in Chapter~\ref{ch:CrossSection}.

Other soft muon tagging techniques exist, these however depend on the presence of a jet in the event to work. The SMT tagger, in its \xsm\ form, only relies on the presence of a muon to measure its performance. The calibration of the tagger on 2012 ATLAS data is presented in Chapter~\ref{ch:Calibration}.

In addition, this means that the tagger can be used to identify muons emerging from the $W$ rather than from semileptonic $b$-decays. The performance of such a technique is studied in Chapter~\ref{ch:Boosted} where the tagger is tested in two ways. Firstly, the tagger is used to identify the muon emerging from the $W$ boson decay. Its performance is compared to the nominal approach, as well as a novel method specifically designed for boosted top searches known as mini-isolation. Secondly, the tagger is used to identify the $b$-jets in the event and its performance in this regime is compared to the standard MV1 tagger.
