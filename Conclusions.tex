% !TEX root = Thesis.tex
\chapter{Conclusions}\label{ch:conclusions}

This thesis explored alternative method for $b$-tagging and muon selection known as the \xsm-based soft muon tagger. The SMT tagger exploits the quality of the match between the ID track and the MS track to tag soft muons produced in the decay of $b$-quarks. The tagger was calibrated on 2012 ATLAS data and used as part of two \ttbar\ measurements. The first was the measurement of the SM \ttbar\ production cross section in the lepton plus jets channel at \cmsS\ using the \xsm-based SMT tagger. The second tested the viability of using the \xsm-tagger in the realm of boosted \ttbar\ searches at \cmsE.

The calibration of the \xsm-based SMT tagger was carried out using low-\pt\ muons from \jpsi\ decays on 2012 ATLAS data at \cmsE. A tag and probe method was used to construct a pool of muons on which to measure the efficiency of the SMT tagger. The efficiency of the tagger monitored as a function of the angular position, the isolation, and the transverse momentum of the candidate muons. No dependence was observed on the isolation of the muon, this makes the calibration on the isolated \jpsi\ muons applicable to \ttbar\ events. No dependence was observed with respect to the azimuthal angle. As in the previous calibration, a dependence on the transverse momentum is observed, as well as a dependence on the pseudorapidity which is asymmetric between the positive and negative sides of the detector. The distribution of the STACO combined \xsd\ appears to be mismodelled in simulation compared to data. This results in a scale factor that deviates from unity by as much as \SI{20}{\percent}. The discrepancy in \xsd\ appears to originate from a mismodelling of the transverse impact parameter or the correlated polar angle. The effects of the detector alignment description on the \xsd\ were noted. Differences between the 2011 and 2012 simulation change the distribution significantly. A more thorough examination of the alignment effects needs to be conducted to determine if this is the source of the discrepancy.

The SM \ttbar\ production cross section at \cmsS\ has been measured in the lepton plus jets channel using the \xsm-based SMT tagger. The multijet background component in the electron channel was measured using the matrix method and the ABCD method. The results of both methodologies are in agreement within their uncertainties. The cross section measured agrees with the latest theoretical predictions and results from other ATLAS and CMS measurements. The soft muon tagger contributes an uncertainty which is comparable to or smaller than other taggers used in other ATLAS \ttbar\ cross section measurements.

The performance of the SMT tagger in a boosted environment has been measured in a preliminary study using only simulated data. Using the tagger to select muons from $W$ bosons yielded some additional acceptance to the mini-isolation approach, however the increase in fake rate makes this methodology less advantageous, particularly since a dedicated treatment of the background would have to be devised.

The SMT tagger appears to function well as a $b$-tagger in a boosted environment. Using the SMT tagger alongside the MV1 yields an \SI{10}{\percent} increase in the number of tagged $b$-jets compared to the MV1 tagger alone. If the uncertainties due to the tagger remain similar to those estimated in the SM \ttbar\ cross section measurement, it would be possible to repeat the resonant \ttbar\ search using the SMT tagger instead. This would require a more thorough examination of the data/MC discrepancy noted in the calibration, or the use of another variable in lieu of the \xsd. The latter is now being explored using the momentum imbalance which exhibits a similar performance and is better modelled in simulation.
