% !TEX root = ../Thesis.tex
\begin{abstract}
  This thesis presents top quark measurements where $b$-jets are identified by searching for ``soft'' muons produced within them. This method, a form of soft muon tagging, discriminates between $b$-jets and jets from other quark types, by using the quality of the match (\xsm) between the muon tracks recorded in the inner detector and muon system of the ATLAS detector.
  The data/MC efficiency scale factor is obtained on ATLAS data at \cmsE\ using a tag and probe method on muons from \jpsi\ decays. The number of muons which are selected by the \xsm-tagger is obtained from a fit to the invariant mass of the pair.
  A measurement of the top quark pair production in the lepton plus jets channel using the soft muon tagger on ATLAS data at \cmsS\ is presented. The multijet background component was estimated using data-driven methods known as the matrix method and the ABCD method. The measured cross section is in good agreement with theoretical calculations and other measurements from ATLAS and CMS\@. The final measured cross section is:
  %
  \begin{equation*}
    \sigma_{\ttbar} = 165\pm2\stat\pm17\syst\pm3\lumi\si{\pico\barn}
  \end{equation*}
  
  The viability of using the \xsm-based soft muon tagger in the search for boosted resonant production of \ttbar\ pairs via the theoretical \Zprime\ boson is also presented. Due to the large boost in the event, the products of the top quarks merge in a collimated cone. The performance of the \xsm-tagger in identifying the $W$ muon and as a $b$-tagger is tested. It is found that the tagger provides an additional acceptance to the $W$ muon of \SI{8}{\percent} over the current method known as mini-isolation. As a $b$-tagger the \xsm-tagger adds an extra \SI{12}{\percent} more $b$-jets when compared to using the MV1 tagger only.
\end{abstract}