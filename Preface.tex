% !TEX root = ./Thesis.tex
\thispagestyle{empty}
\vspace{1cm}
\chapter*{Preface}
This thesis describes various top measurements performed using a novel method, referred to as \emph{soft muon tagging} (SMT), for identifying the decay of $b$-quarks by tagging the muons produced from the semilpetonic decay of these quarks. The implementation of soft muon tagging used here relies on the quality of the match between tracks in the inner detector and muon systems of the ATLAS detector. In addition, the calibration of this methodology is also described here.
Chapter~\ref{ch:Theory} includes an introductory overview of the Standard Model of Particle physics. Chapter~\ref{ch:TopQuark} includes a more detailed description of top quark physics, including the production mechanisms and decay modes; the experimental signature of top events at hadron colliders; and some of the latest results in the field of top quark measurements. Chapter~\ref{ch:Detector} includes: a description of the ATLAS detector and all its components relevant to the study of the top quark, including the inner detector and muon systems; a short introduction to particle physics event simulation; and object reconstruction techniques used at ATLAS including the SMT tagger. The measurement of the data/simulation SMT efficiency scale-factor on 2012 ATLAS data is detailed in Chapter~\ref{ch:Calibration}. The measurement of the top quark pair production cross section using the SMT tagger was performed and is detailed in Chapter~\ref{ch:CrossSection}. Chapter~\ref{ch:Boosted} includes a feasibility study measuring the potential performance of the SMT tagger in the search for theoretical particles that produce pairs of top quarks with very high momentum.

The calibration presented in Chapter~\ref{ch:Calibration} is based on a standard method for calibration widely used in the ATLAS collaboration. The object selection used are based on a previous calibration performed by a former member of the RHUL top quark group. This selection was however adapted to work with 2012 ATLAS data and completely reimplemented by me using up-to-date software tools and a different type of data-sample. All results, plots and/or diagrams presented are my own unless otherwise noted. The cross section measurement presented in Chapter~\ref{ch:CrossSection} is the result of the joint RHUL-QMUL work group and includes contributions from current and past members of the group. The multijet background estimation in the electron channel using data-driven techniques was contributed by me and is described in more detail in Section~\ref{sec:CrossMultijetElectron}. I have contributed a chapter detailing this estimation to the published paper in Ref.~\cite{Cross:SMTCrossSectionPaper}. Finally, Chapter~\ref{ch:Boosted} includes a comparison between the SMT tagger and a lepton identification technique known as mini-isolation. This technique was devised and developed by other members of the ATLAS collaboration, however the performance measurement presented here are my own work. Once again all results, plots or diagrams in this chapter are my own unless stated otherwise.
