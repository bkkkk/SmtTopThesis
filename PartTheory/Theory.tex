% !TEX root = ../Thesis.tex
\newcommand\scalemath[2]{\scalebox{#1}{\mbox{\ensuremath{\displaystyle #2}}}}
\chapter{The Standard Model of particle physics}\label{ch:Theory}

Particle physics is the study of the fundamental constituents of matter and their interactions. The best current description of these interactions is known as The Standard Model of Particle Physics (SM); a group of theories that cover all currently known particles and their interactions. The SM was developed throughout the latter half of the twentieth century and has stood the test of time and rigorous examination by numerous experiments. Many of its parameters have also been measured with great precision e.g.\ the electron magnetic moment $g$ is known to \num{e-13}~\cite{Theory:AwesomeSM}. The last piece to be confirmed was the existence of the Higgs boson, which in turn points to the existence of the so-called Higgs field. Evidence of this particle was observed by the ATLAS and CMS experiments at CERN in 2013~\cite{Theory:HiggsDiscoveryATLAS,Theory:HiggsDiscoveryCMS}.
Despite its tremendous success, the SM cannot explain all observed phenomena in the universe. Firstly, the theory does not predict the value of all of its parameters and many of them, like the number of particle generations, must be measured empirically. The theory also does not describe gravity the most familiar of the fundamental forces. Furthermore, the SM does not provide a candidate for dark matter or dark energy, which according to recent measurements accounts for more than \SI{90}{\percent} of the total energy density in the universe~\cite{Theory:DarkMatter}. The clear asymmetry between matter and antimatter is also not fully explained in the realm of the SM.\@ Because of these deficiencies there is a strong focus on developing theories which go beyond the standard model (BSM) to provide an answer to these open questions.

In this chapter an introductory overview of the Standard Model is provided. For a more detailed description of the theory see references~\cite{Theory:Perkins,Theory:IntroGriffiths} on which this chapter is largely based.

The SM describes the interactions of the fundamental constituents of our universe in terms of the three different fundamental forces: the strong, weak and electromagnetic (EM), each described by a specific theory. The most familiar of the forces, gravity is not described. The SM classifies particles into several categories depending on their properties and allowed interactions. Particles which have a half-integer spins (e.g. $S=\frac{1}{2}$, $\frac{3}{2}$,\ldots) are known as \emph{fermions}, these are the basic constituents of matter. Particles with integer spins (e.g. $S=0$, 1,\ldots) are known as \emph{bosons}, these mediate interactions between fermions and other bosons.

Fermions can be divided into two subgroups: quarks, which can interact via the strong, weak and electromagnetic forces; and leptons which can only interact via the weak and electromagnetic forces. There are six known leptons: electron $e$, muon $\mu$ and tau $\tau$, which all have electric charge\footnote{The electric charge is always stated in units of elementary charge $e$} $Q=1$; and the corresponding electrically neutral neutrino $\nu_e$, $\nu_\mu$ and $\nu_{\tau}$. Analogously, six quark \emph{flavours} are known: $u$, $c$ and $t$, with electric charge $Q=+2/3$ and $d$, $s$ and $b$, with electric charge $Q=-1/3$.

Quarks and leptons are divided into three generations which differ only by the mass and flavour of their constituent fermions, each generation being heavier than the previous. A summary of all elementary particles described by the SM can be found in Table~\ref{tab:TheorySmParticles}.

For every matter fermion $f$ there is an equivalent antimatter partner $\bar{f}$ which possesses the same characteristics as its matter companion but is opposite in electric charge. Thus 12 matter particles are combined with 12 antimatter partners for a total of 24 elementary particles which form all visible matter in the universe.

Interactions between fermions occur via the exchange of spin one particles known as bosons. As shown in Table~\ref{tab:TheoryForces}, each force is mediated by one or more bosons. The strong force is mediated by a set of massless bosons known as the gluons, the weak by a neutral massive boson known as the \Z\ boson and a pair of charged massive bosons known as the \W\ bosons. Finally, the electromagnetic force is mediated by the massless photon. Each boson has an antimatter partner however some, like the photon, are indistinguishable from their matter version. A summary of the properties of the SM bosons is shown in Table~\ref{tab:TheorySmParticles}.

Each fermion has a set of so-called quantum numbers which classify the type of interactions that can occur. For example, each lepton has a \emph{lepton number} associated with it, electrons have an electron lepton number $L_e=+1$, while the positron has $L_e=-1$. Muons and taus have their own respective lepton numbers, $L_{\mu}$ and $L_{\tau}$. Each neutrino has lepton number $L_{f}=1$ and their anti-matter counterpart have $L_f=-1$. Each of these lepton numbers is approximately conserved separately across interaction vertices. The conservation is only approximate due to the non-zero mass of neutrinos. Another example of a quantum number is \emph{baryon number} $B$. Each quark has $B=\frac{1}{3}$ and antiquarks have $B=-\frac{1}{3}$.

%% Particle Table
\begin{table}[p]
  \centering
    \includegraphics[width=0.95\textwidth]{PartTheory/Diagrams/ParticleTable.pdf}
    \caption[A summary of all elementary particles described by the SM]{A summary of all elementary particles described by the SM~\cite{Theory:PDGBooklet}. Note the various groupings and divisions including by spin, generation and particle type. For each particle the charge ($q$), mass and name are shown as per the legend on the bottom-right.}\label{tab:TheorySmParticles}
\end{table}

%% Table of Forces
\begin{table}[htbp]
  \centering  
    \begin{tabular}{@{}lll@{}}
      \toprule
      Name            & Relative Strength & Boson \\
      \midrule
      Strong          & \num{e38}         & Gluons \\
      Electromagnetic & \num{e36}         & Photon \\ 
      Weak            & \num{e25}         & $W$ and $Z$ \\
      Gravity         & \num{1}           & Graviton* \\
      \bottomrule
    \end{tabular}
    \caption[A summary of the four fundamental forces ordered by relative strength.]{A summary of the four fundamental forces ordered by approximated relative strength. These are included to demonstrate the large differences in strength that span many orders of magnitude. A more accurate determination of the interaction strength depends on the details of the interaction itself. $^*$ The graviton is the theoretical boson responsible for mediating gravitational interactions and is not part of the SM.}\label{tab:TheoryForces} 
\end{table}

\section{Quantum electrodynamics}

The interaction of particles via the electromagnetic force is described by \emph{quantum electrodynamics} (QED). These interactions are mediated by the massless neutral boson known as the photon and the strength of the interaction is characterized by the fine-structure constant $\alpha$. All electrically charged fermions are allowed to interact and since the photon itself is not charged, no self-interaction is allowed within QED\@. Figure~\ref{fig:TheorySimpleQED} shows the single vertex described by QED where two fermions interact via a photon. Note that the electric charge is conserved across the vertex, so for example $\gamma\rightarrow e^{+}e^{+}$ is not allowed within QED\@.

%% Simple QED Vertex
\begin{figure}[htbp]
  \centering
    \begin{fmffile}{simpleqed}
\begin{fmfgraph*}(100,80)
\fmftop{fi,fo} \fmfbottom{Photon}
\fmf{fermion}{fi,vx1,fo}
\fmf{boson,label=$\gamma$}{vx1,Photon}
\fmflabel{$f$}{fi} \fmflabel{$f$}{fo}
\fmfdot{vx1}
\end{fmfgraph*}
\end{fmffile}
    \caption[The fundamental interaction vertex described by QED.]{The fundamental interaction vertex described by QED\@. The straight-lines represent any charged fermion, while the wavy line is a photon. All possible QED vertices can be obtained by simply rotating this vertex.}\label{fig:TheorySimpleQED}
\end{figure}

By combining different forms of this vertex one can build every possible QED interaction. The interaction $e^{+}e^{-}\rightarrow e^{+}e^{-}$ is known as Bhabha scattering. Two leading order (LO) diagrams contribute to this interaction, annihilation (Figure~\ref{fig:TheoryQEDTreeA}) and scattering (Figure~\ref{fig:TheoryQEDTreeB}). More complex diagrams with two additional vertices, those with loops for example, are said to be at next to leading order (NLO). Even more complex diagrams of the same process are at NNLO and so on.

%% Example Tree Level Diagrams
\begin{figure}[htbp]
  \centering
    \begin{minipage}[][][t]{.47\textwidth}
      \centering
        \begin{fmffile}{TheoryQEDTreeA}
  \fmfframe(3,12)(1,14) { % left, top, right, bottom
    \begin{fmfgraph*}(100,80)
      \fmfleft{ele1,pos1}
      \fmfright{ele2,pos2}
      \fmf{fermion}{ele1,v1}
      \fmf{fermion}{v1,pos1}
      \fmf{photon,label=$\gamma$}{v1,v2}
      \fmf{fermion}{ele2,v2}
      \fmf{fermion}{v2,pos2}
      \fmflabel{$e^{+}$}{pos1} \fmflabel{$e^{+}$}{pos2}
      \fmflabel{$e^{-}$}{ele1} \fmflabel{$e^{-}$}{ele2}
    \end{fmfgraph*}
  }
\end{fmffile}
        \subcaption{Electron-positron pair annihilation.}\label{fig:TheoryQEDTreeA}
    \end{minipage}
    \,
    \begin{minipage}[][][t]{.47\textwidth}
      \centering
        \begin{fmffile}{TheoryQEDTreeB}
  \fmfframe(3,12)(1,14) { % left, top, right, bottom
    \begin{fmfgraph*}(100,80)
      \fmfleft{ele1,pos1}
      \fmfright{ele2,pos2}
      \fmf{fermion}{ele1,v1}
      \fmf{fermion}{v1,ele2}
      \fmf{photon,label=$\gamma$}{v1,v2}
      \fmf{fermion}{v2,pos1}
      \fmf{fermion}{pos2,v2}
      \fmflabel{$e^{+}$}{pos1} \fmflabel{$e^{+}$}{pos2}
      \fmflabel{$e^{-}$}{ele1} \fmflabel{$e^{-}$}{ele2}
    \end{fmfgraph*}
  }
\end{fmffile}
        \subcaption{Electron-positron pair scattering.}\label{fig:TheoryQEDTreeB}
    \end{minipage}
    \caption[Feynman diagrams of the process $e^{+}e^{-}\rightarrow e^{+}e^{-}$ allowed in QED at leading order.]{Feynman diagrams of the process $e^{+}e^{-}\rightarrow e^{+}e^{-}$ allowed in QED at leading order. Additional vertices can be added to produce higher-order diagrams of the same process.}\label{fig:TheoryQEDTree}
\end{figure}

\section{Quantum chromodynamics}

Interactions via the strong force are described in the theory of \emph{quantum chromodynamics} (QCD). These interactions are mediated by a set of massless neutral bosons known as gluons. QCD introduces the concept of colour which dictates which interactions are allowed via the strong force. Colour can take three states, red (anti-red), blue (anti-blue), green (anti-green):
%
\begin{equation}
  r=\begin{pmatrix}
    1 \\
    0 \\
    0 \\
  \end{pmatrix},
  \qquad
  g=\begin{pmatrix}
    0 \\
    1 \\
    0 \\
  \end{pmatrix},
  \qquad
  b=\begin{pmatrix}
    0 \\
    0 \\
    1 \\
  \end{pmatrix}
  \end{equation}

  \begin{equation}
  \bar{r}=\begin{pmatrix}
    1 &
    0 &
    0 
  \end{pmatrix},
  \qquad
  \bar{g}=\begin{pmatrix}
    0 &
    1 &
    0 
  \end{pmatrix},
  \qquad
  \bar{b}=\begin{pmatrix}
    0 &
    0 &
    1 
  \end{pmatrix}
\end{equation}

Both quarks and gluons possess colour and as a result gluons can self-interact in a three gluon vertex (Figure~\ref{fig:TheoryQCDThreeGluon}) or a four gluon vertex (Figure~\ref{fig:TheoryQCDFourGluon}). As with electrical charge, colour-charge must also be conserved. In the scattering process $q\rightarrow qg$, shown in Figure~\ref{fig:TheoryQCDColour}, the flavour of the quark does not change but the colour-charge does. The difference in colour is carried away by the scattered gluon. Thus each gluon has two colour states associated with it, a colour state and an anti-colour state. Naively one would expect nine different types of gluons that participate in interaction, because of the nine combinations of colour and anti-colour. However the $\textrm{SU(3)}$ symmetry on which QCD is based results in a colour octet:
%
\begin{equation}
  \begin{aligned}[c]
    &(r\bar{b}+b\bar{r})/\sqrt{2} \\
    -i&(r\bar{b}-b\bar{r})/\sqrt{2} \\
    &(r\bar{r}+b\bar{b})/\sqrt{2} \\
    &(r\bar{g}+g\bar{r})/\sqrt{2}
  \end{aligned}
  \qquad\qquad
  \begin{aligned}[c]
    -i&(r\bar{g}-g\bar{r})/\sqrt{2} \\
    &(b\bar{g}+g\bar{b})/\sqrt{2} \\
    -i&(b\bar{g}-g\bar{b})/\sqrt{2} \\
    &(r\bar{r}+b\bar{b}-2g\bar{g})/\sqrt{6}
  \end{aligned}
\end{equation}
%
and the overall colourless ``colour singlet'':
%
\begin{equation}
  (r\bar{r} + g\bar{g} + b\bar{b})/\sqrt{3}
\end{equation}

There are eight different gluons that can participate in interactions each with a different colour-charge combination, and a ninth colourless gluon that does not interact. Gluons being colour-charged has far reaching consequences for QCD\@.

In the realm of QED the vacuum around an electric charge becomes polarized as opposite charges get attracted and like charges are repelled. This has the effect of partially cancelling out the electric field experienced at a finite distance from the central charge. This effect is known as screening and also occurs with colour-charge. Quark-antiquark pairs screen the true colour-charge of the central real quark.

However, since gluons also carry colour they cause the opposite effect (anti-screening) to amplify and change the observed colour of the quark. Which effect, screening or anti-screening, wins out depends on the number of colours in the theory and the number of quark flavours. Currently, three colour states and six different quark flavours are known. This makes screening the overall dominant effect and as a result, the colour potential decreases with distance and quarks experience very little potential when very near to each other. This phenomenon is known as asymptotic freedom and forces quarks to form bound colourless states known as \emph{hadrons}.

Hadrons can be divided into two categories: \emph{mesons}, which contain a quark and an antiquark ($q\bar{q}$); and \emph{baryons}, which are made of three anti/quarks each with a different anti/colour-charge to result in a colourless composite particle. Common examples of baryons are protons ($uud$) and neutrons ($udd$) which are the building blocks of atomic nuclei. The pion $\pi^{0}=u\bar{u}/d\bar{d}$ is a meson which is commonly produced in hadron colliders. Due to their quark configuration, baryons have baryon number $B=+1$ while mesons have $B=0$.
  
%% Self-interacting QCD Gluon vertices
\begin{figure}[htbp]
  \centering
    \begin{minipage}[][][t]{.32\textwidth}
      \centering
      \begin{fmffile}{ColourQCD}
\fmfframe(5,17)(20,17) {
\begin{fmfgraph*}(100,80)
\fmftop{qin,qout} \fmfbottom{glu}
\fmf{quark}{qin,vertex1,qout}
\fmf{gluon,label=$g$,l.d=10}{vertex1,glu}
\fmfdot{vertex1}
\fmflabel{$q$}{qin} \fmflabel{$q$}{qout}
\end{fmfgraph*}  
}
\end{fmffile}
      \subcaption{Quark-gluon vertex}\label{fig:TheoryQCDColour}
    \end{minipage}
    \begin{minipage}[][][t]{.32\textwidth}
      \centering
        \begin{fmffile}{selfqcd3}%
\fmfframe(5,17)(20,17) {
\begin{fmfgraph*}(100,80)
\fmfleft{glu1}
\fmfright{glu3,glu4}
\fmf{gluon}{vertex1,glu1}
\fmf{gluon}{glu3,vertex1} \fmf{gluon}{glu4,vertex1}
\fmfdot{vertex1}
\fmflabel{$g$}{glu1}
\fmflabel{$g$}{glu3}
\fmflabel{$g$}{glu4}
\end{fmfgraph*} }
\end{fmffile}
        \subcaption{Three-gluon vertex}\label{fig:TheoryQCDThreeGluon}
    \end{minipage}
    \begin{minipage}[][][t]{.32\textwidth}
      \centering
        \begin{fmffile}{selfqcd4}
  \fmfframe(1,10)(1,10) { %
    \begin{fmfgraph*}(100,80)
      \fmfleft{glu1,glu2}
      \fmfright{glu3,glu4}
      \fmf{gluon}{vertex1,glu1} \fmf{gluon}{vertex1,glu2}
      \fmf{gluon}{glu3,vertex1} \fmf{gluon}{glu4,vertex1}
      \fmfdot{vertex1}
      \fmflabel{$g$}{glu1} \fmflabel{$g$}{glu2}
      \fmflabel{$g$}{glu3} \fmflabel{$g$}{glu4}
    \end{fmfgraph*} 
  }
\end{fmffile}
        \subcaption{Four-gluon vertex}\label{fig:TheoryQCDFourGluon}
    \end{minipage}  
    \caption[The fundamental interaction vertices described by quantum chromodynamics.]{The fundamental interaction vertices described by quantum chromodynamics. Shown are \subref{fig:TheoryQCDColour} gluon emission from a quark, \subref{fig:TheoryQCDThreeGluon} gluon emission from a gluon, and \subref{fig:TheoryQCDFourGluon} the four-gluon vertex.}\label{fig:TheoryQCDVertexes}
\end{figure}

\section{Weak interactions}\label{sec:TheoryWeakInteractions}

The final type of interaction involves the so-called weak force. The weak force is responsible for common nuclear processes such as for $\beta^{-}$ decay ($n\rightarrow pe^{-}\bar{\nu}_{e}$). Interactions via the weak force are mediated by three massive bosons: the neutral $Z^{0}$ boson and the $W^{+}$ and $W^{-}$ bosons. Since these posses mass the range of interaction is very short, unlike electromagnetic interactions via a massless photon.

All fermions can interact via the weak force, but to start let's consider weak interactions involving only leptons. A valid interaction via the weak force occurs via a combination of the fundamental vertices shown in Figure~\ref{fig:TheoryWeakVertexes}, while conserving electric charge and lepton flavour. Weak bosons also couple to each other via the vertex $Z\rightarrow W^{-}W^{+}$. As the $W$ bosons have charge they also couple to the photon.

%% Weak Neutral Vertexes
\begin{figure}[htbp]
  \centering
    \begin{minipage}[][][t]{.32\textwidth}
      \centering
        \begin{fmffile}{WeakNeutral}
\begin{fmfgraph*}(100,80)
\fmftop{fi,fo} \fmfbottom{ZBoson}
\fmf{fermion}{fi,vx1,fo}
\fmf{boson,label=$Z^{0}$}{vx1,ZBoson}
\fmflabel{$f$}{fi} \fmflabel{$\overline{f}$}{fo}
\fmfdot{vx1}
\end{fmfgraph*}
\end{fmffile}
        \subcaption{Neutral vertex}\label{fig:TheoryWeakNeutralFermions}
    \end{minipage}
    \begin{minipage}[][][t]{.32\textwidth}
      \centering
        \begin{fmffile}{WeakCharged}
  \fmfframe(5,10)(6,10) { % left, top, right, bottom 
    \begin{fmfgraph*}(100,80)
      \fmftop{fi,fo} \fmfbottom{WBoson}
      \fmf{fermion}{fi,vx1,fo}
      \fmf{boson,label=$W$}{vx1,WBoson}
      \fmflabel{$\ell$}{fi} \fmflabel{$\nu_{\ell}$}{fo}
      \fmfdot{vx1}
    \end{fmfgraph*}
  }
\end{fmffile}
        \subcaption{Charged vertex (leptons)}\label{fig:TheoryWeakChargedLeptons}
    \end{minipage}
    \begin{minipage}[][][t]{.32\textwidth}
      \centering
        \begin{fmffile}{WeakChargedQuark}
  \fmfframe(5,10)(6,10) {%
    \begin{fmfgraph*}(100,80)
      \fmftop{fi,fo} \fmfbottom{WBoson}
      \fmf{fermion}{fi,vx1,fo}
      \fmf{boson,label=$W$}{vx1,WBoson}
      \fmflabel{$\bar{q}$}{fi} \fmflabel{$q'$}{fo}
      \fmfdot{vx1}
    \end{fmfgraph*}
  }
\end{fmffile}
        \subcaption{Charged vertex (quarks)}\label{fig:TheoryWeakChargedQuarks}
    \end{minipage}
    \caption[The fundamental interaction vertices described by the weak theory.]{The fundamental interaction vertices described the weak theory. Shown are the \subref{fig:TheoryWeakNeutralFermions} neutral vertex, \subref{fig:TheoryWeakChargedLeptons} charged vertex with leptons, and \subref{fig:TheoryWeakChargedQuarks} charged vertex with quarks. Where $f=e\textrm{, }\mu\textrm{, }\tau$ and $\nu_{\ell}$ is the corresponding lepton neutrino of the same flavour.}\label{fig:TheoryWeakVertexes}
\end{figure}

Weak interactions involving quarks are more complicated than those with only leptons. The neutral vertex is similar to that of the leptonic version, a quark scattering off a \Z\ boson. However, the charged current changes the flavour of an up-type quark into a down-type quark (or vice-versa) with an associated \W\ boson of the appropriate charge (Figure~\ref{fig:TheoryWeakChargedQuarks}). This change in flavour can also happen across quark generations. The semileptonic decay of $b$-quarks is an example of flavour changing charged weak interactions. The $b$-quark (in a $B$ meson bound state) transitions into a $c$-quark by emitting a $W$ boson. In order to account for such an interaction and preserve the universality of weak interactions, Nicola Cabibbo postulated~\cite{Theory:CKMNicola} that the states that couple to the charged current are really a mixture of `rotated' quark states:
%
\begin{equation}
  \begin{pmatrix}
    u \\
    d' \\
  \end{pmatrix}
  \begin{pmatrix}
    c \\
    s' \\
  \end{pmatrix}
\end{equation}
%
where
%
\begin{subequations}
  \begin{equation}
  \label{eq:TheoryWeakQuarkMixingEq1}
  d'=d\cos\theta_{c} + s\sin\theta_{c}
  \end{equation}
  \begin{equation}
  \label{eq:TheoryWeakQuarkMixingEq2}
  s'=-d\sin\theta_{c} + s\cos\theta_{c}
  \end{equation}
\end{subequations}

This introduces an arbitrary parameter into the theory known as the quark mixing angle or the Cabibbo angle $\theta_{c}$. The introduction of quark mixing has the effect of attenuating the interaction strength at vertices involving multiple quark generations. Interactions which cross one generation are said to be Cabibbo Suppressed while those that cross two generations are Doubly Cabibbo suppressed.

Taking into account the three quark generations, quark mixing can be expressed in matrix notation as shown in Equation~\ref{eq:TheoryWeakQuarkMixingMatrix}. This unitary matrix is known as the Cabibbo-Kobayashi-Maskawa matrix (CKM matrix) after Cabibbo who initially postulated quark mixing, and Makoto Kobayashi and Toshihide Maskawa who later added an additional generation, containing the top and bottom quarks, to the matrix~\cite{Theory:CKMKobayashiMaskawa}. The interaction strength at a given vertex is then proportional to $|V_{ij}|^{2}$.

\begin{equation}
  \label{eq:TheoryWeakQuarkMixingMatrix}
  \begin{pmatrix}
    d' \\
    s' \\
    b' \\
  \end{pmatrix}
  =
  V_{CKM}
  \begin{pmatrix}
    d \\
    s \\
    b \\
  \end{pmatrix}
  =
  \begin{pmatrix}
    V_{ud} & V_{us} & V_{ub} \\
    V_{cd} & V_{cs} & V_{cb} \\
    V_{td} & V_{ts} & V_{tb} \\
  \end{pmatrix}
  \begin{pmatrix}
    d \\
    s \\
    b \\
  \end{pmatrix}
\end{equation}

Several parametrizations of the CKM matrix exist, the Chau-Keung parametrization~\cite{Theory:ChauKungCKM} uses angles $\theta_{\textrm{12}}$, $\theta_{\textrm{23}}$, $\theta_{\textrm{13}}$ and a phase $\delta$:

\begin{equation}
  \label{eq:TheoryWeakCKMStandard}
  V_{CKM}
  =
  \scalebox{0.85}{\mbox{
  \ensuremath{\displaystyle \begin{pmatrix}
    c_{12}c_{13} & s_{12}c_{13} & s_{13}\exp(-i\delta) \\
    -s_{12}c_{23}-c_{12}s_{23}s_{13}\exp(i\delta) & c_{12}c_{23} - s_{12}s_{23}s_{13}\exp(i\delta) & s_{23}c_{13} \\ 
    s_{12}s_{23}- c_{12}c_{23}s_{13}\exp(i\delta) & -c_{12}s_{23}-s_{12}c_{23}s_{13}\exp(i\delta) & c_{23}c_{13} \\
  \end{pmatrix}}
  }}
\end{equation}
%
where $c_{ij}=\cos\theta_{ij}$ and $s_{ij}=\sin\theta_{ij}$ for $i=1$,2,3. This parametrization has the advantage that each angle $\theta_{ij}$ relates to a specific transition from one generation to the other. If $\theta_{13} = \theta_{23} = 0$ the third generation is not coupled to the other two and the matrix is reduced to the one postulated by Cabibbo. Note that $\theta_{12}$ is the Cabibbo angle described earlier.

Another parametrization due to Wolfenstein~\cite{Theory:CKMWolfenstein} expresses all elements in terms of the Cabibbo angle by defining $\lambda\equiv s_{12}=\sin \theta_{12}$ and then expressing the other elements in powers of $\lambda$
%
\begin{equation}
  V_{CKM}
  \approx
  \begin{pmatrix}
  1-\lambda^2/2 & \lambda & A\lambda^3(\rho-i\eta) \\
  -\lambda & 1-\lambda^2/2 & A\lambda^2 \\ 
  A\lambda^3(1-\rho-i\eta) & -A\lambda^2 & 1\\
  \end{pmatrix}
\end{equation}
%
where $A$, $\rho$ and $\eta$ are all real numbers that express the order of magnitude differences between $s_{12}$ and the other elements in the matrix.

All the elements should be the same irrespective of which parametrization is used. The elements of the CKM matrix have been measured and the latest accepted results~\cite{Theory:PDGBooklet} are summarized in Equation~\ref{eq:TheoryWeakCKM}. 

The unitarity of the CKM matrix implies that the probability of transition from any up-type quark to any down-type is the same,
%
\begin{equation} 
  \label{eq:TheoryWeakMixingTotal}
  \sum_{k}|V_{ik}|^{2}=\sum_{i}|V_{ik}|^{2}=1
\end{equation}
%
for all $i$ quark generations~\cite{Theory:WeakUniversaility}.
The term $V_{tb}$ is approximately unity and by far dominates over the other $V_{tj}$ terms. This means that the top quark transitions almost exclusively into a $b$-quark ($t\rightarrow Wb$) with transitions $t\rightarrow Ws$ and $t\rightarrow Wd$ having a probability of less than $1\%$. The soft muon tagger which is the focus of this thesis relies on weak semileptonic decays of $b$-quarks. From~\ref{eq:TheoryWeakCKM}~\cite{Theory:PDGBooklet} one can see that the transition $b\rightarrow c$ dominates over $b\rightarrow u$. This thesis concerns itself with \ttbar\ events in the lepton plus jets channel where one $W$ boson decays hadronically with a rate governed by the elements of the matrix.

\begin{equation}
  V_{CKM}
  =
  \begin{pmatrix}
    0.97427\pm0.00015 & 0.22534\pm00065 & 0.00351\;\substack{+\;0.00015\\-\;0.00014} \\
    0.22520\pm0.00065 & 0.97344\pm0.00016 & 0.0412\;\substack{+\;0.0011\\-\;0.0005} \\
    0.00867\;\substack{+\;0.00029\\-\;0.00031} & 0.0404\;\substack{+\;0.0011\\-\;0.0005} & 0.999146\;\substack{+\;0.000021\\-\;0.000046} \\
  \end{pmatrix}
  \label{eq:TheoryWeakCKM}
\end{equation}

An additional unique feature of weak interactions is that the charge conjugation-parity ($CP$) symmetry is violated. The operator $C$ denotes the change of a particle by its antiparticle partner and $P$ denotes a spatial inversion. A clear violation of $C$ and $P$ was observed in the radioactive decay of Cobalt-60, where the resulting electrons were preferentially emitted in the opposite direction of the nuclear spin of the Cobalt~\cite{Experimentalb}. Thus weak currents only couple to left-handed neutrinos (or right-handed anti-neutrinos) which is a violation of parity. Additionally charge symmetry is also violated since a left-handed neutrino is preferentially picked over a left-handed anti-neutrino. Finally in 1964 $CP$ violation was observed in the decay of neutral kaon~\cite{Evidence}.

Thus the probability of $\bar{a}\rightarrow \bar{b}$ is not equal to that of $a\rightarrow b$. The existence of $CP$ violation has interesting consequences for the formation of the early universe. The preferential production of matter over antimatter in $CP$ violating interactions would shift the balance in favour of matter resulting in a universe similar to our own. In terms of the Wolfenstein parametrization of the CKM matrix, if $\eta=0$ there is no $CP$ violation. This parameter has been measured to be non-zero pointing to $CP$ violation~\cite{Theory:PDGBooklet}.

\subsection{Electroweak unification and the Englert-Brout-Higgs mechanism}

The unification of the electromagnetic and weak theories was first proposed by Glashow and later developed by Weinberg and Salam into the electroweak theory~\cite{Model,Theory:WeakInteractionsGlashow,Theory:WeakEMInteractions}. The theory postulates that while at low energies the two forces are to be treated separately, at higher the two can be seen as a single force. Thus the two forces are different manifestation of the same ``electroweak'' interaction. There were several stumbling blocks to the unification of the forces. Firstly, the boson which drives the electromagnetic interaction, the photon, is massless while the weak bosons are both massive. Evidence for the massive nature of these bosons has been established by experimental results from the UA1 experiment at CERN~\cite{Theory:WBosonObservationPaper}.

Thus the symmetry of the theory must be spontaneously broken in some way. A mechanism for electroweak symmetry breaking (EWSB) was proposed and developed by Anderson, Brout, Englert, Higgs, Guralnik, Hagen, Kibble, and t' Hooft which introduces masses to the weak bosons and posits the existence of an additional scalar (spin $S=0$) boson known as the Higgs boson.

\subsubsection{Gauge theories}

Gauge invariance is one of the underlying invariances which underpins the Standard Model. Given the so-called Dirac Lagrangian\footnote{A Lagrangian is a mathematical function that describes the underlying dynamics of a system as a function of time and space coordinates ($x^{\mu}$) and their time derivatives.}
%
\begin{equation}
  \label{eq:TheoryHiggsDiracLagrangian}
  \Lagr = i\hbar c \bar{\psi}\gamma^{\mu}\partial_{\mu}\psi -m^2\bar{\psi}\psi
\end{equation}
%
which describes a free particle of spin-$\frac{1}{2}$ with mass $m$~\cite{Theory:IntroGriffiths}. Note that it is invariant under the transformation
%
\begin{equation}
  \psi\rightarrow e^{i\theta}\psi
\end{equation}
%
where $\theta$ is a real number, since the adjoint $\bar{\psi}\rightarrow e^{-i\theta}\bar{\psi}$ and the two terms cancel out. This is known as a \emph{(global) gauge transformation} since $\theta $ is the same at all points of space-time. A \emph{(local) gauge transformation} occurs when the phase is different for different points in space-time
%
\begin{equation}
  \psi\rightarrow e^{i\theta(x)}\psi
\end{equation}

The Dirac Lagrangian in Eq.~\ref{eq:TheoryHiggsDiracLagrangian} is not invariant under a local gauge transformation since extra terms are created by the derivative. This then implies that the underlying physics of such a theory depends on position in space-time. Thus local gauge invariance must be imposed. In the case of the Dirac Lagrangian, this is done by introducing additional terms to the Dirac Lagrangian which will cancel the extra terms introduced by the local gauge transformation. As it turns out this results in the introduction of a new massless vector field that couples to $\psi$.

The new Lagrangian then describes a spin-$\frac{1}{2}$ particle with mass $m$ that interacts with a free massless field. This new field can be identified as the electromagnetic field and the spin-$\frac{1}{2}$ particles are electrons and positrons. Thus the resulting Lagrangian describes all interactions that form part of quantum electrodynamics.

A similar procedure can be applied to the colour quark model and obtain a description of all QCD interactions. However requiring that the weak theory be a gauge theory (invariant under local gauge transformation) encounters a problem since the weak bosons are known to be massive. There must be some mechanism via which the $W^{\pm}$ and $Z^{0}$ obtain mass.

\subsubsection{The Englert-Brout-Higgs mechanism}

The Englert-Brout-Higgs mechanism\footnote{Here the ATLAS naming convention is used.} posits the existence of a complex scalar field doublet that when introduced into the electroweak Lagrangian results in the weak fields acquiring a mass term. In other words the $W^{\pm}$ and $Z^{0}$ interact with the Higgs field and obtain a mass. An additional consequence of introducing the Higgs field is the inclusion of a scalar boson particle, the so-called ``Higgs boson''. Finally, the Higgs field also couples to fermions via the Yukawa coupling generating gauge invariant mass terms for the fermions as well\footnote{For a more complete description of the mathematical procedure see~\cite{Theory:IntroGriffiths}.}. This coupling is dependent on the mass of the fermion involved, for a more massive particle the coupling is stronger. This is another reason for the top quark being an object of much study.

The SM Lagrangian in its current form including the Higgs potential is shown in Equation~\ref{eq:TheorySMLagrangian}. This expression describes all possible particle interactions that form part of the SM, of particular interest are the fermion mass term which couples the fermion field $\psi$ to the scalar Higgs field $\phi$ and the Higgs kinetic and potential terms.

\begin{align}
  \Lagr = &- \underbrace{ \frac{1}{4} W^a_{\mu\nu} W^{\mu\nu a} }_{ \text{Weak Field} }
           - \underbrace{ \frac{1}{4} B_{\mu\nu} B^{\mu\nu} }_{ \text{EM Field} }
           - \underbrace{ \frac{1}{4} G^a_{\mu\nu} G^{\mu\nu a} }_{ \text{Strong Field} } \nonumber \\
          &+ \underbrace{ \bar{\psi}\slashed{D}_{\mu}\psi}_{ \text{Fermion Kinetic} }
           + \underbrace{ \lambda\bar{\psi}\psi\phi }_{ \text{Fermion Mass} } \label{eq:TheorySMLagrangian} \\
          &+ \underbrace{ |D_{\mu}\phi^2| }_{ \text{Higgs Kinetic} }
           - \underbrace{ V(\phi) }_{ \text{Higgs Potential} } \nonumber
\end{align}

The Higgs boson, and consequently the EBH mechanism, was the last remaining piece of the SM that resisted experimental confirmation. In late 2012, the ATLAS and CMS collaborations announced~\cite{Theory:HiggsDiscoveryCMS,Theory:HiggsDiscoveryATLAS} the discovery of a Higgs-like particle with a mass around \SI{125}{\GeV}~\cite{Theory:HiggsMass}, confirming the last missing component of the SM\@. However, the remaining unexplained phenomenon have yet to be theoretically described and experimentally confirmed. Due to its large mass, the top quark is of much interest to BSM searches.
